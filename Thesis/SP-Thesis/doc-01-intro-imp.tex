\chapter{Úvod}
\label{introduction}

Videoherní průmysl, ačkoliv je stále relativně mladý, představuje
jednu z nejvýdělečnějších a nejdůležitějších forem médií v moderní době.
Nejenže se stále více integruje do kultury, ale již v roce 2018 byl
tento průmysl schopen vygenerovat 134,9 miliardy dolarů\cite{VideoGameProfit}, a toto číslo každým rokem stále roste.

S tímto růstem a úspěchem však přichází určitá forma amnézie. 
Většina firem, které se zabývají tvorbou a publikováním videoher, se soustředí pouze na moderní systémy. 
Ty hry, které byly v minulosti odpovědné za vývoj a růst tohoto průmyslu, rychle mizí z dějin lidské kultury, podobně jako pára nad hrncem.

\subsection{Dostupnost her}

V roce 2023 byla provedena studie\cite{VideoGameSurvery} společností \textit{Video Game History Foundation}, 
zaměřující se na dostupnost starých videoher na moderních systémech, 
a to zda-li je vůbec možné zakoupit staré videohry oficiálně a legální cestou.

Výsledek tohoto výzkumu byl skličující. 
Z 4000 vytipovaných videoher vydaných před rokem 2010 je nedostupných \textbf{87\%}\cite{VideoGameSurvery}, 
a není možné je získat oficiální legální cestou (studie se zabývala pouze obchody v USA). 
Článek tvrdí, že existuje několik důvodů, proč je historie videoherního průmyslu v takto zanedbaném stavu.

Jedním z důvodů jsou technické problémy, 
kdy portování videohry z jednoho systému na jiný může být netriviální záležitost 
a vyžaduje alokaci finančních zdrojů. Dalším a důležitějším faktorem nedostupnosti 
videoher je problém licenčních práv, kde samotný zákon ztěžuje dostupnost 
a samozřejmě také distribuční prostředky mohou v tomto problému hrát roli.

I přesto, jak je tento problém široce pochopen, není mu přisuzována téměř žádná váha v širších kruzích.

\subsection{Emulátory}

Většina způsobů, jak získat možnost hrát historické videohry, ve většině případů hraničí
se zákonem. Člověk je nucen nelegálně stahovat software z internetových stránek,
které jsou dedikované pro protizákonému sdílení licencovaného obsahu.

Ovšem to je možné pouze tehdy, pokud daná videohra byla vytvořena pro hardwarový systém,
který daný člověk již vlastní. Jestliže byla videohra publikována pouze na platformě,
která se již 20 let neprodává a není již nijak oficiálně podporována, má člověk smůlu.
To platí i tehdy když si člověk oficiálně hru zakoupil, ale poté nemá dostupný hardware, který by hru
dokázal spustit.

Problém nedostupnosti samotných her je obtížně řešitelný, avšak problém nedostupnosti
historického hardwaru je na tom podstatně lépe. Ačkoliv herní konzole jsou
speciálně navrženy a tedy patentovány, nic veřejnosti nebrání vytvořit softwarový simulátor,
který se snaží co nejpřesněji napodobit chování hardwaru herních konzolí a
zároveň umožnit určitou videohru si zahrát, jak ukázal soudní spor mezi
firmami Sony Computer Entertainment a Connectix Corporation\cite{SonyVsConnectix}.
Implementace emulátoru jedné staré herní konzole je náplní této práce.
