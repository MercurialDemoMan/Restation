\chapter{Závěr}

Ačkoliv tento emulátor nemá ani vzdáleně 100\% kompatibilitu s celou \textit{PlayStation} knihovnou, rozhodně jsem rád za to, 
že se mi podařilo dostat emulátor do stavu, kde nejenom tento software je schopen nastartovat originální \textit{PlayStation} hry,
ale pár z nich bylo i relativně hratelných. V tomto projektu šlo hlavně o to prozkoumat
komerční a velmi populární systém a pokusit se zpříjemnit jeho používání pro komunitu herních nadšenců milující staré hry, 
ať už kvůli zvědavosti, nebo kvůli nostalgii.

Implementace takto složitého systému vyžaduje přehlédnutí určitých specifických chování hardwaru, přeskočení logiky fungování
komponentu, nebo dokonce přeskočení celistvé komponenty. Čili nebudu zastírat, že tu a tam chybí cihla či trám. Tento nedostatek
je způsoben tím, že mohu čerpat pouze z dokumentů, které jsou výsledkem mnohaleté práce reverzního
inženýrství mnoha lidí a ačkoliv se jedná o monumentální úsilí, nakonec nikdo (krom \textit{Sony}) nemá přístup k celému originálnímu
hardwarovému návrhu \textit{PlayStation} konzole. 

Následující výčet popisuje, jak schopné jsou hlavní hardwarové komponenty tohoto emulátoru a poskytuje hrubý procentuální odhad dokončení:

\begin{itemize}
    \item{\textbf{CPU} \textbf{91\%} - Chybí řádné zpracování výjimek a CPU nepodporuje hardwarové breakpointy}
    \item{\textbf{GPU} \textbf{80\%} - Ačkoliv samotná funkcionalita je plně pokryta (kromě kreslení polyčar), \textit{GPU} má spoustu chyb a nepřesností (fixed-point aritmetika, či nepřesná rasterizace polygonu).}
    \item{\textbf{MDEC} \textbf{65\%} - \textit{MDEC} v emulátoru podporuje pouze 16-bitovou a 24-bitovou barevnou hloubku, ale originální hardware umožňuje dekódovat 4-bitovou a 8-bitovou barevnou hloubku (\textit{grayscale}).}
    \item{\textbf{CDROM} \textbf{42\%} - Z celkových \textbf{38} příkazů, \textbf{20} z nich je implementováno a na zbylých \textbf{18} emulátor padá. \textit{CDROM} také dokáže pouze načítat disky s 1 stopou.}
    \item{\textbf{GTE} \textbf{70\%} - Ze \textbf{23} příkazů, \textbf{17} z nich funguje. Nicméně je to komponenta, ze které pramenilo nejvíce chyb a implementačních problémů.}
    \item{\textbf{DMA} \textbf{80\%} - Přenosy mezi komponentami fungují, ale synchronizace je oproti reálnému hardwaru velmi zjednodušená.}
    \item{\textbf{Časovače} \textbf{95\%} - U časovače typu \textit{DotClock} je jeho rychlost aproximovaná, jinak by vše mělo fungovat.}
    \item{\textbf{Periférie} \textbf{50\%} - Simulace digitálního ovladače funguje, ale emulátor nemá podporu pro paměťové karty pro uložení postupu ve hře.}
    \item{\textbf{SPU} \textbf{5\%} - \textit{SPU} implementuje pouze svůj hardwarový interface aby ostatní komponenty s ním mohly komunikovat, ale interně se v něm neděje prakticky nic.}
\end{itemize}

Samotná změna rozlišení, až na pár artefaktů ohledně adresování textur, funguje relativně dobře. U změny rozlišení je hlavní problém ten, že zásadně zpomaluje chod emulátoru jak bylo popsáno v naměřených datech \ref{cpu-performance}. Nejrychlejší z testovaných strojů zvládal s rezervou emulovat systém s \textbf{2x} násobkem rozlišení, ale při \textbf{4x} násobku již nebyl schopen dosáhnout požadované rychlosti. Jak bylo zmíněno v sekci [\ref{gpu-optimalization}], řešením tohoto zpomalení by bylo přesunout vykreslování na hostovací \textit{GPU}.