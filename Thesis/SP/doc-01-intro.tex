\chapter{Úvod}
\label{introduction}

Videoherní průmysl, ač stále mlád, je jednou z nejvýdělečnejší a 
nejdůležitější formou média moderní doby. Nejenom že se čím dál víc
integruje do kultury, ale jen v roce 2018 tento průmysl byl
schopen vygenerovat \$134.9 miliard dolarů a
toto číslo každým rokem pouze narůstá.

S takovým růstem a úspěchem ovšem přichází amnézie. Většina firem,
zabývající se tvorbou a publikováním videoher, se zaměřuje pouze
na moderní systémy a videohry. Ty videohry za které v minulosti byly zodpovědné,
se vypařují z dějin lidstva rychleji než pára nad hrncem.

\subsection*{Dostupnost her}

V roce 2023 byla provedena studie společností \textit{Video Game History Foundation}, 
zaměřující se na dostupnost starých videoher na moderních systémech, 
či zda-li je vůbec možnost staré videohry oficiálně a legálně zakoupit.
Výsledek tohoto výzkumu by skličující. \textbf{87\%} ze \textbf{4000} vytipovaných videoher
vydané před rokem \textit{2010} jsou nedostupné a není možné je
jakkoliv získat oficiální a legální cestou (studie se zabývala pouze
obchody ve Spojených Státech). Článek navrhuje, že důvodů proč je historie
videoherního průmyslu v takto zbědovaném stavu je několik.

Jednak to jsou technické problémy, kde portování videohry z jednoho systému
na jiný může být netriviální záležitost a vyžaduje alokaci finančních zdrojů.
Ale dalším a důležitějším faktorem nedostupnosti videoher je problém licenčních práv,
kde samotný zákon ztěžuje dostupnost, přičemž samotné distribuční prostředky mohou
také hrát roli.

I přesto, jak pochopený tento problém je, nadává se mu skoro žádné váhy v
širších kruzích.

\subsection*{Emulátor}

Většina způsobů jak mít historické videohry dostupné ve většině případů hraničí
se zákonem. Člověk je nucen nelegálně stahovat software z internetových stránek,
dedikované na protizákoné sdílení licencovaného obsahu.

Ovšem to je možné jedině tehdy, pokud daná videohra byla vytvořena pro hardwarový systém,
který daný člověk již vlastní. Jestliže videohra byla publikována pouze na platformě,
která se již 20 let neprodává a není nijak podporována, má člověk smůlu.
To platí i tehdy, když člověk si oficiálně hru zakoupil, ale pak nemá dostupný hardware.

Problém nedostupnosti samotných her je problém těžko řešitelný, ovšem problém nedostupnosti
historického hardwaru je na tom podstatně lépe. Ačkoliv herní konzole jsou
speciálně navrženy a tedy patentovány, nic veřejnosti nebrání vytvořit softwarový simulátor,
který se snaží co nejpřestněji napodobit chování hardwaru herních konzolí a 
zároveň umožnit určitou videohru si zahrát, tak jak ukázal soudní spor mezi 
firmami \textit{Sony Computer Entertainment} a \textit{Connectix Corporation}.

Implementace emulátoru jedné staré herní konzole je náplní této práce.
